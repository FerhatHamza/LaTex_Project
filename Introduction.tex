\chapter*{\centering Introduction Général} 
	\paragraph{}
	Notre civilisation à produit plus d’information durant ces trente dernières années, que pendant les cinq milles ans qui les ont précédés. Et selon les analystes on devrait en disposer du double dans les cinq prochaines années.
	\paragraph{}
	De nos jours, le Web représente une source de données et d’informations interrogée par un grand nombre d’internaute. Les requérant du Web sont de profile très variés et ont donc des objectifs différents et partagent de nombreux services disponibles sur le Web ; par exemple : les bibliothèques numériques, les systèmes d’information online, les journaux numériques.
	\paragraph{}
	Notre projet de fin d’étude consiste à mettre en place un site web dynamique pour une Virtual Academy, permettant essentiellement la mise à la disposition des utilisateurs des vidéo et des documents   classées par catégories tout en laissant la possibilité d’ajouter d’autres vidéos ou documents.
	\paragraph{}
	Par référence à cette présentation, nous adopterons la structure générale suivante de notre travail :
	
	\begin{itemize}
		\item Le premier chapitre est consacré à une présentation générale de la technologie du Web.
		\item Le deuxième chapitre présente les langages de programmation pour la conception des sites web, ainsi que les bases de données.
		\item Le troisième chapitre concerne le développent du système Virtual Academy, il décrit essentiellement son architecture et son implémentation.
		\item Le quatrième chapitre poursuit un cas d’application de notre système.
		\item Et enfin nous terminerons par une conclusion, ou nous évoquerons les principaux apports de notre travail.
	\end{itemize}